\chapter{Introdução}

O campo de pesquisa conhecido como \gls{ia}, reconhecido como um dos domínios mais enigmáticos e promissores da ciência contemporânea, originado a partir dos avanços em Ciência de Dados e do \gls{am}, tem se consolidado como uma das áreas mais influentes dos últimos tempos. Seu impacto transcende a Tecnologia da Informação, promovendo transformações em campos como a medicina, a educação e, notoriamente, o direito --- algo que será explorado ao longo deste trabalho.

A magnitude de seu impacto é tamanha que figuras proeminentes da indústria tecnológica vêm se pronunciando de forma audaciosa sobre suas expectativas em relação a essa tecnologia emergente. Sundar Pichai, CEO da Google, declarou que "a inteligência artificial é mais profunda do que a eletricidade ou fogo", enfatizando o potencial disruptivo dessa tecnologia IA.

Após períodos de estagnação, denominados "Invernos da Inteligencia Artificial" \cite{russell_artificial_2016} --- marcados pela frustração de expectativa e pela retração de investimentos --- o campo tem experimentado, nas últimas décadas, um crescimento exponencial. Esse avanço tem se destacado especialmente no subcampo conhecido como \gls{pln}, notoriamente com o advento dos \gls{llm}s, revolucionando a forma como sistemas computacionais interagem com a linguagem humana. Entre os modelos mais notáveis estão os da família GPT (Generative Pre-trained Transformer) e o LLaMA (Large Language Model Meta AI), ambos classificados como \gls{llm}s de propósito geral.

A eficiência dos LLMs na manipulação da linguagem natural deve-se, em grande parte, à arquitetura Transformers, proposta por \citeonline{vaswani_attention_2017}. Essa arquitetura é baseada no mecanismo de atenção, mais especificamente na chamada self-attention, que possibilita a análise contextual de cada elemento textual (token), independente da sua posição na sequência. Assim, cada token é processado não apenas considerando seu significado isolado, mas também seu contexto em relação aos outros tokens de entrada.  

Esse modelo de processamento é crucial para tarefas de transdução de sequência, ou sequence-to-sequence, nas quais uma sequência de entrada é convertida em uma sequência de saída. Trata-se de uma estrutura essencial para aplicações avançadas no campo do PLN, como tradução automática, reconhecimento de fala, sumarização de textos, análise de sentimentos, geração de texto e reconhecimento de entidades nomeadas. A versatilidade dessa abordagem tem possibilitado avanços significativos nos sistemas de automação, permitindo que tarefas historicamente atribuídas à cognição humana possam ser assumidas por sistemas computacionais.

Essa capacidade já vem sendo explorada em diversas áreas que lidam com uma vasta quantidade de dados e informações, muitas vezes na forma de documentos contendo texto não estruturado. No âmbito jurídico, ferramentas como DraftWise têm desenvolvido soluções baseadas em IA para auxiliar advogados na redação e negociação de contratos.

Além disso, estudos como o de \citeonline{mahoney_framework_2019} propõem frameworks para classificação de texto aplicável no contexto da revisão de documentos legais, visando melhorar a eficiência e a transparência na identificação de documentos relevantes durante processos legais.   

Notoriamente, o processo de documentação e confecção de documentos a partir de texto não estruturado, uma tarefa extremamente relevante para a área do direito, se alinha muito bem com as capacidades dos modelos de LLM e suas afinidade com tarefas de transdução de sequências. Escritórios e departamentos jurídicos elaboram diariamente documentos como petições, procurações, contratos, acordos e atas de reunião, uma tarefa de natureza complexa, agravada pelo alto fluxo de trabalho que qualifica a prática jurídica.

Portanto, diante desse contexto, a automação da redação de documentos jurídicos por meio de técnicas de NLP associadas a Modelos de Linguagem de Grande Escala mostra-se como uma solução promissora para solução programática para o desafio descrito. Essa abordagem visa não apenas otimizar o desempenho das atividades jurídicas, mas também promover eficiência, padronização e redução de erros em tarefas que demandam processamento intensivo de linguagem natural.  

\section{Motivação}
\label{sec:motivacao}

No contexto político e judiciário, a transparência e a audibilidade configuram-se como pilares essenciais para garantir a confiabilidade das informações e a continuidade na gestão dos processos institucionais (Fernandes, 2021). Tais práticas não apenas sustentam a integridade das ações administrativas e jurídicas, como também promovem a devida prestação de contas à sociedade, elemento central em uma democracia.

Nesse cenário, as atas de reunião assumem um papel estratégico como instrumento formal de registro. Sua importância reside na capacidade de sintetizar, de forma clara, precisa e organizada, os acontecimentos relevantes de um encontro institucional, incluindo os temas debatidos, as decisões tomadas e os encaminhamentos definidos. Diferentemente de uma simples transcrição literal das falas, a ata oferece uma representação estruturada dos eventos, o que facilita sua posterior consulta e contribui diretamente para a governança e a memória organizacional.

Além da sua função administrativa e documental, as atas também desempenham um papel crucial no cumprimento das normas legais de transparência e acesso à informação. Sua elaboração e disponibilização pública estão em conformidade com legislações como a Lei de Acesso à Informação \cite{brasil_lei_2011} e a Lei da Transparência \cite{brasil_lei_2009}, reforçando o compromisso institucional com a responsabilidade e o controle social.

Apesar de sua relevância, a produção de atas ainda é, majoritariamente, uma tarefa manual, morosa e sujeita a inconsistências de estilo, conteúdo e qualidade. A ausência de padronização entre diferentes órgãos e setores acentua esses desafios, dificultando a verificação, a interoperabilidade e o uso eficiente desses documentos. Diante desse cenário, torna-se evidente a necessidade de soluções que automatizem e otimizem esse processo, promovendo ganhos de tempo, padronização e conformidade legal.

\section{Objetivos}

\label{sec:objetivos}


%Estudar a aplicação de técnicas de Processamento de Linguagem Natural (NLP), Modelos de Linguagem de Grande Escala (LLMs) e abordagens derivadas da Inteligência Artificial na automação do processo de síntese de atas de reunião.     

\subsection{Objetivo Geral}
\label{sec:objetivo-geral}


O objetivo principal deste trabalho é desenvolver um software baseado em LLMs de código aberto para automatizar a elaboração de atas de reunião, avaliando sua eficácia e aplicabilidade nesse contexto. A proposta envolve a criação de uma solução capaz de processar transcrições estruturadas e extrair automaticamente informações essenciais, como: lista de participantes e seus respectivos cargos ou funções, pautas deliberadas, unidades proponentes, pautas extraordinárias sugeridas durante a reunião, assuntos discutidos e deliberações finais.


\subsection{Objetivos Específicos}
\label{sec:objetivos-especificos}

Com essa abordagem, busca-se promover maior agilidade, precisão e padronização na elaboração de atas, contribuindo para a transparência e a eficiência dos processos institucionais. Essa iniciativa torna-se especialmente relevante em contextos nos quais a documentação tem papel crítico, como nos ambientes jurídico e administrativo. Para alcançar o objetivo geral, o trabalho será orientado por três objetivos específicos: 

	\begin{alineas}
		\item  Investigar o uso de modelos avançados de inteligência artificial na automação da extração de informações a partir de transcrições de reuniões
		\item Analisar o impacto de diferentes configurações dos modelos e de prompts sobre a qualidade das respostas geradas
		\item  Testar e comparar o desempenho de diversos modelos de linguagem natural em condições controladas, avaliando critérios como precisão, consistência e adequação às necessidades do domínio jurídico.
	\end{alineas}

Ao final da pesquisa, espera-se atingir dois resultados principais: (i) gera conclusões quanto ao desempenho dos modelos de LLM utilizados na tarefa de automação de Atas de reunião e como pode se aplicar para outros documentos; e (ii) a entrega de uma aplicação funcional, configurada como um produto mínimo viável (MVP), que integre todos os componentes desenvolvidos e comprove a viabilidade técnica e prática da solução. A aplicação deverá apresentar uma arquitetura escalável, eficiente e com potencial de uso real em ambientes institucionais.